\chapter{INTRODUCTION}
%%%%%%%%%%%%%%%%%%%%%%%%
% The grand intro, ancient greeks, life as we know it, etc.
%%%%%%%%%%%%%%%%%%%%%%%%
Water is one of the most important molecules found on our Earth. 
It occurrs in natural abundance, composing nearly XX percent of the molecules
found. Some hypothesize that life cannot exist without water, as we have
not found a case of life without it. In many organisms, water comprises XX
percent of their chemical makeup, with is large as XX in others. In humans,
water composes XX percent of our bodies. 

Even in ancient times, humans recognized the importance of water. Indeed
it was one of the four elements described by the Greeks (Plato?), 
alongside earth, 
wind, and air. The importance of water was discovered early in our human 
history, and it has been studied extensively since those early days.
As we have studied water, we have found that water breaks most trends in
properties, including XX, XX, as well as, the negative
slope in the solid-liquid coexistence line in the pressure-temperature
phase diagram. Perhaps most famously, the density of water decreases through 
the transition of liquid to solid, unlike any other molecule we know of.XX
Indeed it is this difference in densities that many attribute to have helped
life flourish, that water in lakes will freeze from the top down, instead of 
from the bottom of the lake to the surface. This has allowed for life to exist 
under the frozen ice. 

%%%%%%%%%%%%%%%%%%%%%%%%
% A bit on bulk liquid water, properties and some chronological ordering
%%%%%%%%%%%%%%%%%%%%%%%%
The study of liquid water has received much attention over the years, its 
structure and behaviors have been analyzed by experiments and more recently 
computer simulations. Talk about a lot of the properties of liquid water here,
and who / when we discovered them.

%%%%%%%%%%%%%%%%%%%%%%%%
% Bulk ice
%%%%%%%%%%%%%%%%%%%%%%%%
Likewise, bulk ice has been studied extensively. The crystal structure of ice
is well known, though we have recently found more possibly structures. There
are 16 known ice polymorphs, and several amorphous crystal structures as well.
X-ray scattering, neutron scattering, etc. 

%%%%%%%%%%%%%%%%%%%%%%%%
% Ice/water interface
%%%%%%%%%%%%%%%%%%%%%%%%
More recently, the properties of the ice/water interface has been studied. 


%%%%%%%%%%%%%%%%%%%%%%%%
% Friction at ice/water interfaces, explain the slider problem 
% Break down into the three contributions of friction
%%%%%%%%%%%%%%%%%%%%%%%%
Tribological studies of ice in contact with a wide variety of materials
(including other ice) has been the focus of research for many years. 
Approximately 150 years ago, Faraday attributed the freezing of two
pieces of ice together to be due to the ice surfaces being covered
with a quasi-liquid layer (QLL).\cite{Faraday1859} This marked the beginning
of the modern investigation on the properties of ice and the role
the surface liquid-like layer plays in ice friction. Soon thereafter,
Thomson incorrectly attempted to explain the presence of the liquid-like layer
as a result of pressure melting.\cite{Thomson1859} Reynolds followed 
Thomson's work, and systematically investigated sliding on 
ice. He also concluded that pressurized melting of the ice surface
was the governing physical process for the obeserved small
coefficients of friction.\cite{Reynolds1901} This view was widely 
accepted, until Bowden and Hughes proposed that frictional 
heating may primarily be responsible for the small coefficients of 
friction observed for ice.\cite{Bowden1939} 

Since these inaugural investigations of ice, a large and diverse
community of scientists has formed, studying ice and ice friction
for applications including ice skating and winter 
sports\cite{Rosenberg2005,Kietzig2010}, 
road safety and shoe soles\cite{Roberts1981,Higgins2008}, 
glacier movement\cite{Casassa1991, Sukhorukov2013, Pritchard2012},
and the fracture of the arctic sea 
ice\cite{Schulson2004,Weiss2007,Feltham2008,Lishman2011,Lishman2013}.  
Experimentally, the surface of ice has been probed by atomic force
microscopy
(AFM)\cite{Doppenschmidt1998,Bluhm1999,Bluhm2000}, scanning force
microscopy\cite{Bluhm1998}, ellipsometry\cite{Beaglehole1980,Beaglehole1993},
nuclear magnetic resonance (NMR)\cite{Ishizaki1996}, X-ray 
diffraction\cite{Dosch1996}, and photoelectron
spectroscopy\cite{Bluhm2002}. Further investigation has been performed by
computer simulations, studying bulk 
ice\cite{Kerr1988,Tse1988,Hayward1997,Gao2000,Rick2005,Dong2001,Weber1983,Wang2005,Kuo2005,Buch1998,Rick2001,Gay2002}, 
ice / vapor\cite{Kroes1992,Devlin1995,Ikeda-Fukazawa2004,Picaud2006,Conde2008,Pereyra2009},
and ice / water\cite{Baez1995,Bryk2002,Bryk2004,Bryk2004a,Gao2000,GarciaFernandez2006,Hayward2002,Hayward2001,Karim1988,Karim1987,Karim1990,Louden2013,Nada1997,NadaH.andFurukawa1995,Nada1996,Nada2000,Nada1997a} interfaces. 

Both experiments and computer simulations point towards the existence of a
QLL forming on the surface of ice at temperatures below
the bulk melting point. The formation of this layer is believe to be
driven by the termination of the periodic crystal structure at the
surface. The surface molecules are only weakly bound to their lattice
positions by the underlying ice, and with appreciable thermal energy
these molecules reorient (and at warmer temperatures translate along
the surface) to maximize their hydrogen bonds. This results in the
formation of the QLL, which is generally accepted as the reason why
ice displays a low friction 
coefficient\cite{Malenkov2009,Dash1995,Rosenberg2005,Dash2006}.


There have been
extensive investigations on ice friction, attempting to ellucidate the
roles of temperature\cite{Roberts1981,Higgins2008,Bowden1939,Evans1976,Derjaguin1988,Liang2003}, sliding
speed\cite{Evans1976,Derjaguin1988,Liang2003}, applied load\cite{Buhl2001,Bowden1939,Derjaguin1988,Baurle2006,Oksanen1982},
contact area\cite{Bowden1939,Baurle2007}, and
moisture\cite{Calabrese1980}. Recently, 
Kietzig \textit{et al.} have performed experiments consisting of sliding 
different steel alloy rings over a prepared ice surface.\cite{Kietzig2009} 
They investigated the effect of surface nanopatterning, hydrophobicity, and 
surface structure of the ice-exposed slider on the ice/slider friction. 
These properties were studied over a wide 
range of temperatures and sliding velocities. They found at all temperatures 
investigated, increasing the slider velocity with constant temperature, 
decreases the friction coefficient. After passing through a minimum, the 
friction coefficient slightly increases. This slight increase was attributed 
to added drag due to capillary bridges forming between the melt film and 
slider. They observed a decrease in friction with increasing temperature, 
up to a minimum at about $-4$ 
degrees Celsius. At temperatures warmer than this, there was an observed 
increase in friction which was attributed to capillary bridges and viscous 
shearing of the melt film. 
Through use of laser irradiation, the slider hydrophobicity was tuned
without changing the chemical nature of the material. Kietzig
showed that laser induced hydrophobicity resulted in fewer capillary 
bridges forming between the slider and the melt film. This reduced the amount
of viscous shearing of the ice-melt, resulting in a lower 
friction coefficient. 
%, resulting in a delayed onset of the increase 
%in friction coefficient with slider velocity.
While most investigations of ice friction focus on heterogeneous
materials\cite{Bowden1939,Evans1976,Derjaguin1988,Liang2003,Liang2005,Baurle2006,Baurle2007,Kietzig2009,Kietzig2010},
there have also been advances made on understanding ice-ice friction\cite{Oksanen1982,Kennedy2000,Maeno2004,Fortt2007,Fortt2011,Lishman2011,Samadashvili2013}.

From these studies, three distinct friction regiemes
have been found depending on the temperature and sliding velocity of the 
material; boundary friction, mixed friction, and hydrodynamic 
friction.\cite{Bhushan2002,Persson2015,Tuononen2016,Kietzig2009,Kietzig2010} 
Under each regieme, the observed 
friction is the result of a different physical process. In boundary friction, 
the lubricating layer of ice melt is only several molecules thick. 
This thin film is unable to support the sliding material's load, and friction
arises due to the surface asperities of the sliding material 
interacting with the ice surface.\cite{Bhushan2002} In the mixed friction 
regime, the ice melt lubricating layer is thicker than in the boundary regime,
but not yet sufficiently thick to maintain the 
sliding material's load. However, the ice melt film reduces solid-solid 
adhesion at the interface. This helps to alleviate some of the frictional 
forces, although the lubricating layer can form capillary bridges with the 
material, resulting in a drag force.\cite{Kietzig2009,Kietzig2010} If an 
ice melt lubricating layer is thick enough
to support the sliding material's load, the material's surface asperities are 
no longer in contact with the surface and the observed friction is primarily 
due to the capillary bridges formed between the ice melt and the material.
Under these conditions, the ice friction is classified as hydrodynamic 
friction.\cite{Kietzig2009,Kietzig2010} Thus the three regiemes are characterized by the 
extent a liquid-like layer of water mitigates the sliding material's load.
Through extension of contact melting theory, Fowler and Bejan have recently 
indicated that the lubricating ice melt film under the sliding material becomes 
thicker toward the trailing end.\cite{Fowler1993} Due to this, as a material 
slides over the surface of ice, the prevailing friction mechanism may be a 
combination of those found in each of the friction regiemes. 

In a recent review\cite{Kietzig2010}, Kietzig \textit{et al.} outlined 
popular experimental techniques used to investigate the coefficients of 
friction for a variety of materials sliding on ice, as well as their 
sensativity to temperature, slider load,
contact area, wettability and hydrophobicity of the slider, and many other
parameters. Of particular interest, the friction coefficients were found to 
increase with increasing slider velocity. This was attributed to three 
physical processes; adhesion forces of the
slider's asperities with the ice surface, breaking of capillary bridges 
between the slider and the ice surface, and the viscous shearing of the 
ice melt across the ice surface. While teasing apart the individual 
contributions has proven challenging, Kietzig\cite{Kietzig2009} and
Persson\cite{Persson2015,Tuononen2016} 
have made significant progress. However, there is still very little known about
water shearing over ice surfaces. Open questions include how the
structure of the interface changes 
during this process, and the role the presented crystal face plays on
the observed friction. 

  

