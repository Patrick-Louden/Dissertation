\chapter{INTRODUCTION}
%% Outline:
%  Facts about water, historic perspective from the ancients
%
%  Describe a water molecule in the gas phase. Next, a collection of
%  condensed liquid water, then ice ice crystal structures, etc.
%
%  Next turn to the anomolies of water, and how some of these have
%  been answered based on what was described about the structure of
%  water in each of its phases.
%
%   One of the anomolies that have been discovered is of the low
%   friction coefficient observed for sliding over ice surfaces,
%   advent of studying ice friction, segway into qll and ice surfaces
%
%  once topic of the thesis is established, introduce tools for
%  studying it, MD, RNEMD, etc.
%
%  Quick summary of how the rest of the dissertation is laid out.
%

% What about the following format
% 1. Broad historic intro to water in general
% 2. Gas phase water (water molecule in isolation)
% 3. Liquid water (water in collections at warm T)
% 4. Ice (water in collections cooled down)
% 5. Ice surfaces, surface premeling, qll
% 6. Friction at ice surfaces
% 7. Molecular Dynamics (in general, e.o.m., periodic boxes, etc.)
% 8. Water models for simulations
% 9. Transport Properties
% 10. Non-equilib. MD attempts to calculate TP
% 11. RNEMD, better at calc. TP



%%%%%%%%%%%%%%%%%%%%%%%%
% The grand intro, ancient greeks, life as we know it, etc.
%%%%%%%%%%%%%%%%%%%%%%%%
Water is the simplest compound of the two most common reactive
elements in the unvierse and it the second most common molecule in the
universe (behind hydrogen, H2). It is composed of two atoms of
hydrogen, and one atom of oxygen, and it has been found to be
fundamental to star formation.  Water is one of the most important
molecules found on our Earth.  It occurrs in natural abundance,
composing nearly XX percent of the molecules present. It is also one
of the few molecules that can be found in its solid, liquid, and
gasseous forms, occurring at ambient temperatures and pressures.  Some
hypothesize that life cannot exist without water, as we have not found
a case of life without it. In many organisms, water comprises XX
percent of their chemical makeup, with as large as XX in others. In
humans, water composes XX percent of our bodies. It is the ubiquitus
nature of water and ice that gives rise to its importance; due to its
natural abundance a wide variety of chemistry and physical processes
occur within this medium, and at its surfaces.

In ancient times, humans recognized the importance of water. The Greek
philosopher Thales of Miletus, one of the seven sages of antiquity
declared that everything was composed of water (around 700 BCE). Later
on in the 5th century BCE, Empedocles described water as being one of
four 'classical' elements of the ancient world, earth, wind, fire, and
water. Even the ancient Chinese described the world around them using
five elements, earth, wood, metal, fire, and water. The importance of
water was discovered early in our human history, and it has been
studied extensively since those early days.  


\subsection{The Structure of H$_\mathrm{2}$O}
%
% description of what water is? structure constitution etc?
% Gas phase molecule in isolation
% Condensed phase liquid water as collection
% Description of ice and it's structures
%
Defining water's structure can prove a bit challenging. In isolation
in the gas phase, water adopts a 'bent' like geometry, with an HOH
bond angle of about 104.5 degrees. The oxygen-hydrogen bond length as
computed by high accuracy quantum mechanical calculations gives 1
\AA~. These bond lengths and bend angles are an equilibrium value, as
they are constantly vibrating. Even at -273.15 degrees Celcius (0
Kelvin) some residual vibrations remain. This is described as a
zero-point energy, and has considerable implications for the potential
energy of the molecules. 

Water's electrons are not distributed about the molecule
symmetrically. This gives rise to higher order multipoles, and it is
these electronic characteristic that give rise to many of water's
characteristic properties. In the gas phase, water has a dipole moment
of about 1.85D. This dipole moment is highly polarizable, that is, it
responds to an external electric field, be it an applied one or the
field created by molecules in its local environment. In the presence
of an external field, the valence electrons in water respond by
separating even more, resulting in a larger dipole moment for the
molecule. In the condensed liquid phase, this value becomes much
harder to probe experimentally, but values have been reported ranging
from 2.2D to 2.9D. 

High accuracy QM calculations on collections of water molecules?

\subsubsection{Liquid Water}
It is this dipole moment polarization in conjunction with hydrogen
bonding which gives rise to water being a liquid at room
temperatures. The strength of the hydrogen bond between two water
molecules is somewhere between the strong covalent bonding
interactions holding the individual water molecules together, and the
weaker van der Waals interaction due to the instantaneous polarization
of molecules electronic clouds of electrons. Numerically, hydrogen
bonds have a strength of about 20 J/mol, about ten times thermal
fluctuation at room temperature. This ratio is important in the
cohesion of molecules into a condensed phase. For water, the order of
magnitude difference between the intermolecular interactions and
thermal fluctuations means that the liquid will not boil at room
temperatures. Intermolecular forces for many other smaller molecules
are not as strong as the hydrogen bonding observed in water. The
intermolecular interactions of H2S are much weaker, for example, and
it boils at -60 degrees Celcius. Most other molecules that weigh on
the order of 18 amu are gasseous at room temperature. The strong
dipole moment of water holds the molecules together in a condensed
phase.

Another interesting property of the hydrogen bond is its
directionality and number of neighbors that bond with it. The proton,
electron distribution, and higher multipole moments that give rise to
the strength of the hydrogen bonds also give rise to the local
structure of liquid water. Water molecules hydrogen bond to a central
molecule forming a tetrahedral configuration. In liquid water,
molecules form local tetrahedral environments Liquid water transitions
to ice when the local ordering becomes long ranged.


\subsubsection{Water Ice}
%
% Describe crystal structures of ice, examples of where some occur.
% Proton disordered structure, residule entropy. 
%
%Annu. Rev. Phys. Chem. 2017. 68:285–304
'Ice is a fundamental solid with important environmental, biological,
geological, and extraterrestial impact.'\cite{Shultz17} 'Ice is
arguablyamongthemostfundamental andubiquitous solids in the Universe
(1–3).Consisting of two of the threemost abundant elements, ice and
its liquid form, water, not only shapeEarth and our Solar System but
also transportmaterial throughout the Universe.' 

'Ice is challenging in part due to configurational flexibility
exemplified by the famous residual entropy of ice at 0K(4).The source
of residual entropy lies in the tetrahedral coordination around the
oxygen atom: Two coordinations are bonds to hydrogen atoms and two are
lone pairs. Hence, water forms an extended network, but there is
considerable flexibility for locating the hydrogen atoms to satisfy
the “ice rules” (5): exactly one hydrogen atommust be located between
any pair of oxygen atoms and every oxygen atom must be covalently
bonded to exactly two hydrogen atoms. Surface termination leads to
dangling valences. Dangling valences, even at the ideal surface, can
be random or patterned (6–10). Finally, as is common with many solids,
the ice surface can relax and/or reconstruct.'

(on qll) 'techniques as diverse as glancing angle X-ray (102, 103),
SFG (82), He atom scattering (104), atomic force microscopy (105,
106), ellipsometry (107), photoelectron spectroscopy (108), nuclear
magnetic resonance (NMR) (109, 110), theory (111–114), and
differential interference contrast (57, 115, 116) conclude that the
QLL exists.' The onset temperature has varied from as cold as 200K to
as warm as 260K, and discrepencies have been attributed to
contaminations to varying probe depths. There is also considerable
disagreement about the nature of the QLL, 'Theory indicates that it is
ordinary supercooled water (117), whereas recent interfacial force
microscopy (105) suggests that the QLL is neither ice nor water but
rather a new viscoelastic phase.' And recently Suzuki \textit{et al.}
have probed the QLL and observed two separate immesible liquid water
flims. 

Section 4.1 deserves further reading, talks about thermodynamics of
the basal, prism, and secondary prismatic crystal surfaces.
%end Annu. Rev. Phys. Chem. 2017. 68:285–304

Water crystallizes into ice in a wide variety of different spacial
arrangements. There are currently 16 known crystal structures of ice,
some of which only occur at large pressure environments and are
thought to be present on the moons of SaturnXX, while others have yet
to be observed but are predicted by computer simulations. Here on
Earth, the most common form of ice is ice-I$_\mathrm{h}$, which has 6
molecules in the unit cell and crystallizes under the P6mm3 space
group.

The structure of ice-I$_\mathrm{h}$ has been studied extensively using
experimental methods such as neutron scattering and x-ray
spectroscopy, as well as NMR and laser spectroscopy.\cite{w. F. Kuhs
  and M. S. Lehmann, in Water Science Reviews 2, edited by F. Franks
  (Cambridge University, New York, 1986),pp. 1-65.} 

From these, a set of 'ice-rules' have been established by Bernal and
Fowler.\cite{} These rules describe the required metrics for accurate
crystals of ice-I$_\mathrm{h}$. Residual entropy due to hydrogen
disordering. Hayward and NAME developed an algorithm for sampling of
ice-I$_\mathrm{h}$ crystals.\cite{HaywardXX}

\subsection{Ice/Water Coexistence}
% General intro to ice/water coexistence
Understanding the ice/water interface and the properties of the
molecules at the interface is important for many areas, including the
process of nucleation and crystal growth or melt. 

% Omar A. Karim and A.D.J. Haymet ChemPhysLett, 138, 6, (1987) 531
In one of the first computer simulations of the ice/water interface,
Karim and Haymet investigated the basal face of an ice-I$_\mathrm{h}$
crystal solvated in bulk water using the TIP4P model of water, where
they observed ice/water coexistence at 240$\pm$5~K. Calculating the
distance across which the oxygen-site density decays from the
twin-peaked ice structures to that of bulk liquid, they estimated the
basal ice/water interface to be approximately 10 \AA~ to 15 \AA~
wide. They further partitioned the system into 3.7 \AA~ wide bins, and
calculated the average self diffusion constant for molecules in each
bin. As the interface was traversed, the diffusion constant decays
from the bulk liquid value to approximately zero.
% End  ChemPhysLett, 138, 6, (1987) 531

% Omar A. Karim and A.D.J. Haymet J.Chem.Phys 89, 11 (1988) 6889
TIP4P basal ice/water interface, larger than previous study. They
compute g(r) for the bulk liquid and show it agrees with recent
neutron scattering data.  While investigating the orientational
structure of the interface, they found that the orientational order
induced by the ice crystal propogates at least one, and possible two
layers deeper into the water than the translational order (ie the
density defined by the oxygen positions). Orientational ordering of
the liquid which from density profiles looks like liquid water, but
orientationally still exhibits perturbations due to the
interface. They observed the same interfacial width using a larger
system (and the same metric of measure, the density of oxygen sites).
% End  J.Chem.Phys 89, 11 (1988) 6889


\subsection{The Surface Premelting of Ice and the Quasi-Liquid Layer}
% 1. Surface cuts of ice-Ih, what do they look like, energetics of
% each. Have image of each surface here, talk about physical structures
% of each, dimensions of channels. Proton ordered surfaces vs proton
% disordered surfaces.
% 2. Driving force for the QLL, not unique to ice. Interesting on ice
% because ice occurs everywhere in the universe, therefore, lots of
% interesting things happen at ice surfaces.
% 3. Experimental and theoretical measures of the QLL, who and how have
% studied it.
% 3a. QLL width
% 3b. QLL structure, dynamics
% 4. Observed oddity of friction on ice surfaces is small, transition to
% next subsection where we will talk about friction studies on ice surfaces.

\subsubsection{Common Surface Faces of ice-I$_\mathrm{h}$}
Include image of the surface features here. Commonly exposed surfaces,
the four I will spend my thesis on, etc.


\subsubsection{Driving Force of the Surface Premelting}
All crystalline materials exhibit a surface premelting at temperatures
approaching their melting point. Molecules at the surface of the
material have fewer neightbor molecules compared to those in the bulk
of the crystal, resulting in a reduced potential. At temperatures
considerably below the materials melting point, these molecules will
reside in their lattice positions as the potential felt by their
underlying lattice and neighbor molecules is sufficiently strong to
prevent thermal fluctuations from dislodging the molecules from their
lattice positions BLAH. However, as temperature increases the internal
vibrations and rotations of the surface molecules increase, until at
some critical temperature (T$^\mathrm{*}$ $<$ T$_\mathrm{m}$), where
the surface molecules will leave their lattice positions and translate
along the surface, attempting to maximize their intermoleculer
interactions with the neihboring molecules and those of the underlying
lattice. These surface molecules are collectively refered to as a
quasi-liquid layer (QLL), as their structure and properties reflect
some of their condensed liquid phase counterparts, but are distinctly
unique.

''the formation of (a QLL) on solid surfaces is now well characterized
theoretically as a premelting surface phase
transition.''\cite{R. Lipowsky, Phys. Rev. Lett. 49, 1575 (1982).}

\subsubsection{Previous Investigations of the QLL}
\subsubsection{QLL Width}
\subsubsection{QLL Structure and Dynamics}

% Arrhenius analysis of anisotropic surface self-diffusion on the
% prismatic facet of ice. Gladich11, PCCP (2011), 13, 19960-19969
Using the six-cite water model of Nada and van der Eerden (NE6),
Gladich \textit{et al.} studied surface diffusion of qll water
molecules on the prismatic surface of an ice-I$_\mathrm{h}$
crystal.\cite{Gladich11} Molecules were determined to be part of the
QLL based on a local tetrahedral order parameter, and only those
molecules considered QLL were incorporated into the calculations. They
investigated diffusion over a wide range of temperatures, from 230K to
287K, which varies from -59K to -2K of undercooling, when compared
with the NE6 model's melting point of 289K. The NE6 model overpredicts
the melting point due to the over structuring of water with the model.

Their results indicated a positive Arrhenius curvature, suggesting the
mechanism of self-diffusion changes with increasing temperature. As
this transition occurs, the energy of activation is also seen to
increase from 29.1 kJ mol$^{-1}$ at low temperatures to 53.8 kJ
mol$^{-1}$ at temperatures close to the melting point. The
self-diffusion is also seen to be anisotropic at low temperatures
(around XX K), and transitions to isotropic around 240-250K. 

Using the local tetrahedral order parameter NOT modified for varying
number of local neighbors. They note that due to this, their estimates
of 

\subsubsection{Anisotropic Diffusion of the QLL}
One dimensional diffusion coefficients (D$_\mathrm{i}$) were computed
via the Einstein's relation\cite{Allen87}
\begin{equation}
D_{i} = \frac{1}{2} \frac{dMSD_{i}(t)}{dt}
\end{equation}
where MSD$_\mathrm{i}(t)$ is the mean square displacements as a
function of time. Gladich \textit{et al.} were careful to exclude any
sublimating water molecules from being included in their calculation,
as these molecules move considerably further distances in time than
their condensed phase counterparts. These MSD$_\mathrm{i}(t)$ were
staggered in starting time by 20~ps, averaged, and from this average
MSD$_\mathrm{i}(t)$ the $D_\mathrm{i}$ were obtained by fitting the
linear portion of the MSD$_\mathrm{i}(t)$ plots, between 5 and 25
ns. However, they argue the obtained $D_\mathrm{i}$ values need
correction, as the molecules residing in the solid ice, and ice-like
molecules within the QLL are incorporated into the calculation of
$D_\mathrm{i}$ through MSD$_\mathrm{i}(t)$ at this point. They assume
only liquid-like molecules in the QLL contribute to the surface
diffusivity, and compute surface diffusion constants
$D^{*}_\mathrm{i}$ according to\cite{Pfalzgraff11}
\begin{equation}\label{D*}
D^{*}_\mathrm{i} = D_\mathrm{i}/Q
\end{equation}
where $Q$ is the mean number of molecules classified as liquid-like
($N_\mathrm{LL}$) divided by the total number of molecules in the
system ($N_\mathrm{Slab}$). They obtain $Q$ by the following simple
ration, where they exclude sublimating molecules ($N_\mathrm{EV}$
since they were removed from the MSD$_\mathrm{i}(t)$ calculations
earlier.
\begin{equation}
Q = \frac{N_\mathrm{LL} - N_\mathrm{EV}}{N_\mathrm{Slab} -
  N_\mathrm{EV}}
\end{equation}
This approach to obtaining the scaling parameter $Q$ improves upon the
method of Pfalzgraff \text{et al.}\cite{Pfalzgraff11}, where they included every
molecule in one ice bilayer, including those that had
sublimated. Using eq. \eqref{D*}, Gladich \textit{et al.} were also
able to compute two-dimensional surface diffusion constants.
\begin{equation}
D^{*}_\mathrm{ij} = (D^{*}_\mathrm{i} + D^{*}_\mathrm{j}) / 2
\end{equation} 

Observing the density of the crystal at partitions transverse to the
interface, Gladich observed the prismatic surface QLL grows
continuously with increasing temperature. At the lowest temperatures
investigated, 230~K, only the outermost bilayer was observed to
participate in the formation of the QLL. At two degrees below the
melting point of the NE6 model, the density profiles indicated that
the two outermost bilayers both are involved in the QLL
formation. This result was similar to those seen by Bishop \textit{et
  al.}, who studied the basal ice surface also using the NE6 water
model. These results also agree with those reported by Conde
\textit{et al.}, who studied QLL on the basal, prismatic, and
secondary prismatic using the SPC/E, TIP4P, TIP4P/Ice, and TIP4P/2005
water models.\cite{Conde08} 

Gladich \textit{et al.} estimated QLL thickness ($\delta$) relating
the number of liquid-like quasi-liquid layer molecule
($N_\mathrm{LL}$) to the number of water molecules in a bulk liquid
with box dimensions of $L_\mathrm{x}L_\mathrm{y}\delta$,
\begin{equation}
\delta =
\frac{N_\mathrm{LL}M}{2\rhoN_\mathrm{A}L_\mathrm{x}L_\mathrm{y}}
\end{equation}
where $M$ is the molar mass of water, $N_{A}$ is Avogadro's number,
and $\rho$ is the density of liquid water; the factor of two accounts for
the two interfaces presented by the QLL simulations. Using the
values of $\rho$ reported by Nada and van der Eerden for supercooled
liquid water with the NE6 potential,\cite{Nada03} Gladich computed
$\delta$ at each temperature investigated, and found the QLL to
increase from 3.2 \AA~ wide at 59K of undercooling to 7.4 \AA~ at two
degrees of undercooling.  

They note a low value of $q$ can be obtained for water molecules
encorporating an amorphous solid, in which water molecules are four
coordinate, but are not structured in a tetrahedral arrangement but
instead in a distorted tetrahedron. 

Conde \textit{et al.} studied surface QLL on the prismatic facet using
the TIP4P/Ice water model,\cite{Conde08} and it was found that the NE6
model systematically predicts a lower QLL thickness, due to the
overstructuring of water. Based on their discrimination of
N$_\mathrm{LL}$ molecules given $q_\mathrm{t}$, the NE6 model was
found to have a larger value for $q_\mathrm{t}$, which would thus
result in fewer molecules being considered QLL.

Gladich \textit{et al.} watched movement of water molecules in the QLL
along each axis independently, and found at low temperatures that
movement normal to the interface happened in concert with large
displacements in the normal plane, even when the normal motion is
still well within the defined QLL. Therefore, these motions transverse
to the interface will largely influence surface diffusion of the
molecules. These motions were in good agreement with the diffusion
mechanism proposed by Bishop \textit{et al.}\cite{Bishop09} and by Bolton
and Pettersson\cite{Bolton00}, which suggested the outtermost molecules of
the QLL moved across a relatively rigid surface. Diffusion
characterized in this way will be highly sensative to the underlying
surface morphology and topography, mainly, if the surface geometry or
potential energy surface is anisotropic, we should expect diffusion
across this surface to also be anisotropic. 

Observations that in-plane diffusion follows motions transverse to the
interface were observed at the warmer temperature as well. Through
this vertical motion, the QLL molecule leaves a well-hydrated local
environment to the outter portion of the surface where there are fewer
hydrogen bond partners. Gladich notes that the activation energy for
diffusion at the warmer temperature might actually be larger than that
for the colder. With increasing temperature and a thicker
surface premelting forming, the differing surface topography is masked
and thus there is no observed anisotropy in surface diffusion.

Plotting the surface diffusion along the two axis independentally
($D^{*}_\mathrm{x}$, $D^{*}_\mathrm{z}$) against inverse temperature,
the activation energy ($E_\mathrm{a}$) for the diffusion was
extracted. Since the curvature of ln$D^{*}$ by inverse temperature is
positive, Gladich concludes the activation energy for the
high-temperature mechanism for diffusion is greater than that of the
low-temperature. They further estimate these values to be 29.1 kJ
mol$^{-1}$ for $E_\mathrm{a}$ the low-temperature (approximately the
energy of on hydrogen bond with the NE6 model, 24.5 kJ mol$^{mol-1}$) and 24.5 kJ
mol$^{-1}$ for that of the high-temperature (roughly two hydrogen bonds). 

Nasello \textit{et al.} investigated surface diffusivity of ice by
observing the formation of grain boundaries on polycrystalline ice
surfaces.\cite{Nasello07} Gladich's computed values for surface diffusivity
agree well with those by Nasello as Gladich's values fall within the
error bars reported by Nasello. Gladich's work agrees well with the
experimental work reported by Price \textit{et al.}\cite{Price99},
especially at warmer temperatures. At cooler temperatures however,
Gladich seems to underestimate surface diffusivity.

It is interesting to note that at warm temperatures, simulations of
supercooled bulk liquid\cite{Picaud06,Mahoney01} have also reproduced surface
diffusivity measured by Price \textit{et al.}. However, the
supercooled bulk liquid simulations predict a negative Arrhenius
curvature, implying the activation energy of diffusion decreases with
increasing temperature, opposite of that observed by Price \textit{et
  al.} and predicted by Gladich \texit{et al.}. Given the difference
in sign for the estimated activation energies, it is clear the
mechanism predicted in each case is drastically different. 

Gladich \textit{et al.} estimated the temperature at which the
anisotropic surface diffusivity becomes isotropic by plotting the
ratio of surface diffusions ($D^{*}_\mathrm{x}$ / $D^{*}_\mathrm{z}$)
by temperature. They observed a transition to about unity between 240K
and 250K, between 49 and 39 K of undercooling for the NE6 model.

% end Gladdich11

\subsubsection{Ice Nanostructures}
%Pan11 Melting the Ice: On the Relation between Melting Temperature
%and Size for Nanoscale Ice Crystals. ACS Nano 5 (2011) 4562-4569
While there has been significant progress made on understanding a
two-dimensional ice surface exposed to vacuum, less is known on the
surface premelting of a three-dimensional ice
nanostructure. Understanding premelting of three-dimensional
nanostructures is of importance as particles in this size range are
commonly found in polar mospheric clouds, at altitudes of 80-90 km,
and are known to scatter light from the ice/water interfaces they
contain.\cite{Murray10} 

Egorov \textit{et al.} studied the sensativity of the meling point
with cluster size, for water clusters from 8 to 40 molecules
large.\cite{Egorov02} They observed a nonmonotonic decrease in the melting
point with growing cluster size, though they concluded this behavior
might be specific to the size range they investigated. (In a follow up
study,) Pereyra and Carignano performed molecular simulations of ice
nanocolumns, where they observed the change in melting point
depression as they varied the rectangular width of the
nanostructure.\cite{Pereyra09} 

Pan \textit{et al.} have recently studied hexagonal ice nanostructures
in the range of 2nm (768 water molecules) to 8 nm (9600 water
molecules) using molecular dynamics simulations with the TIP4P water
model.\cite{Pan11} These nanostructures exposed both basal and primary
prism surfaces, as well as two unique edges; one edge where two prism
surfaces meet at a 120 degree angle, and an edge where a basal and
prism surface meet at a 90 degree angle. These nanostructures also
exposed corners where three surfaces intersected, allowing for
investigation of surface premelting at a variety of low-coordination
sites.

% this feels a summarizey, might take out the next P.
They observed that surface premelting occurred first at the corner
sites, followed by the edge sites, and lastly on the flat surfaces.
In addition, they also observed a strong size dependence in the
melting point depression, agreeing well with the classical
Gibbs-Thomson relation. 

Pan \textit{et al.} observed the nanocrystals ordering decrease with
increasing temperature, where the overall shape of the crystal
transitioned from hexagonal to spherical. Interior water molecules
were found to mantain their tetrahedral ordering up to an undercooling
temperature of about 5K to 2.5K. In general, they observed surface
premelting onset between 30 to 40K below the bulk melting point of the
model, in agreement with NAME and NAME. 

Pan \textit{et al.} observed the potential energy increase with
increasing temperature up to the melting point of the ice crystals,
$T^{c}_\mathrm{m}$; at the melting point of the crystals a first-order
phase transition occurs and the temperature appears to rise
vertically. From the jumps in temperature, Pan was able to estimate
the $T^{c}_\mathrm{m}$ for the ice nanocrystals, and saw
$T^{c}_\mathrm{m}$ decrease with decreasing crystal size. Melting
point depression that decreases with decreasing size for nanoscale
clusters of molecules has previoulsy been described by the
Gibbs-Thomson relation. 
\begin{equation}\label{GibbsThomson}
\Delta T^{c}_{m} = T^{b}_{m} - T^{c}_{m} = \frac{\gamma T^{b}_{m} K}{L}
\end{equation}
Here, $T^{b}_{m}$ is the bulk melting temperature, $T^{c}_{m}$ is the
nanocrystal melting temperature, $\gamma$ is the free energy of the
solid/liquid interface, $K$ is the average curvature of the interface,
and $L$ is the volumetric latent heat of melting. $K$ can be obtained
from the ratio of the change in surface area to the change in volume,
$K = dA/dV$. The accuracy of which the Gibbs-Thomson relation predicts
melting point depression for ice nanocrystals has been unclear in
previous studies,\cite{Makkonen00,Makkonen02,Della02,Campbell02},
however, Pan \textit{et al.} were able to show their computed
temperature depressions matched well with the Gibbs-Thomson equation.

Using the local tetrahderal order parameter, Pan \textit{et al.}
discriminated ice-like and liquid-like water molecules with the
partitioning criteria of ice-like molecules having $q_{i} \geq 0.91$,
and all other molecules being considered liquid-like, following Conde
\textit{et al.}.\cite{Conde08} Using this criteria, Pan computed the
thickness of the QLL by
\begin{equation}
d_{apparent} = \frac{n_{QLL}M_{H_{2}O}}{2\rho N_{A} \pi r^{2}}
\end{equation}
where $n_{QLL}$ is the mean number of liquid-like water molecules
within a cylinder of radius $r = 0.6$~nm, $M_{H_{2}O}$ is the molar
mass of water, $\rho$ is the density of TIP4P at it's melting point
(0.99 g/cm$^{3}$\cite{Bluhm00,Conde08}), and $N_{A}$ is Avogadro's number.
Using this metric, Pan \textit{et al.} estimate the onset temperature
of the QLL (defined as the temperature at which the QLL becomes 0.1 nm
thick on the basal plane\cite{Conde08}), as well as determine the thickness
of the QLL on the exposed basal and prismatic surfaces of the ice
nanocrystals. 

With their definition of onset temperature, they observed bulk
premelting (a comensurate crystal with an exposed basal surface) at
100K below the melting temperature of the model. For the nanocrystals,
they observed an onset temperature at about 90K to 95K below the
melting point of the nanocrystals. 

% for the water model discussion, copied from text
High melting temperatures (>400 K) predicted by widely used density
functional theory (DFT) exchange corelation functionals such as
Perdew-Burke-Ernzerhof (PBE)\cite{Perdew96} and Becke-Lee- Yang-Parr
(BLYP)\cite{Becke88,Sprik96} are a further reason to favor force fields over
DFT.\cite{Yoo09}

%end Pan11 

\subsubsection{The Anomolous Friction of Ice}



For ice-I$_\mathrm{h}$, the temperature at which surface molecule
vibrations become prevelent at has been reported between 200K and
271K. Experimentally the surface molecules have been probed using a
variety of techniques and have determined slightly different
temperatures at which these internal vibrations become prevelent.

Both experiment and theory agree that the thickness of the QLL on ice
increases approaching T$_\mathrm{m}$, and values have been reported
between 2 nm to over 45 nm at 271~K. Part of the reason for the
discrepency is due to the definition of what molecules constitute a
QLL, as well as the probe used to measure the molecules. Most
experimental measurements observed a gradual and continuous increase
in the QLL width with increasing temperature to T$_\mathrm{m}$,
however, simulations by Kreos \textit{et al.} predicted the melting
occurs in a bilayer-to-bilayer manner. This result was recently
confirmed by Sa'nchez \textit{et al.}, where they probed the QLL
molecules using sum frequency generation spectroscopy to probe the
hydrogen-bonded OH stretch frequency between temperatures of 235~K
and 273~K, the temperature range where many agree the surface
premelting occurs at.

% PRL 117, 096101 (2016) Summary
% Premelting-Induced Smoothening of the Ice-Vapor Interface
Benet \textit{et al.} computer simulations of the QLL of ice fromed on
the primary prismatic facet of an ice I$_\mathrm{h}$ crystal at the
ice-vapor interface close to the ice I$_\mathrm{h}$-liquid-vapor
triple point of water.\cite{Benet17} They found that close to the
triple point(2K below it), the QLL on the prismatic face behaves as
two independent interfaces, the ice-water interface and at further
distances normal to the surface, the water-vapor interface. This
result was prominent for small wavelengths, but observing over large
wavelengths the two interfaces smooth into one gradual interface at
long wavelengths.

Ice crystal morphologies grown from bulk vapor change from plates, to
columns, to plates, and back to columns as the temperature is cooled
down below the triple point, and dendrites can form at high
saturations.\cite{K. G. Libbrecht, Rep. Prog. Phys. 68, 855 (2005).}
The varying crystalline morphologies are attributed to the crossover
in growth rates of the basal and prismatic faces, however, we do not
currently understand what structural transformation at the surface to
drive the crossover in growth rates.\cite{1,2,4,5} Kuroda and Lacmann
have attributed the crossover in crystal growth rates to be due to the
fact that QLL formation occurs at different temperatures for different
exposed crystal facets.\cite{6} (Name and Name have determined QLL
formation to occur at T1 and T2 for basal and prismatic respectively,
where the higher energy pyramidal and secondary prism surface exhibit
premelting at T3 and T4.)

Experimental observations of the QLL on ice\cite{8-14}, though the
presence of impourities has a very large impact on the surface
structure.\cite{12,15} Due to the large perturbations made by these
impurities, and the vast array of probes used to study the QLL,
properties such as the premelting temperature and thickness of the QLL
is still highly debated.\cite{8} 

Benet \textit{et al.} used the TIP4P/2005 water model. Interrogating
water molecules to determine if they are a part of the QLL or that of
ice was performed using the q$_{6}$ order parameter\cite{42}, which
was later optimized to differentiate betwee icelike and waterlike
molecules via correlations of the second nearest neighbors.\cite{43}
Using this parameter, Benet observed an average QLL thickness of
0.9~nm, agreeing with experimental observations\cite{12,14} and
computer simulations.\cite{38-40} Using the q$_{6}$ order parameter,
the ice-liquid and liquid-vapor interfaces were determined, and each
surface was Fourier transformed to give the spectrum of surface
fluctuations.

Analyzing the Fourier transformed surfaces, they found that for QLL
films less than 1 nm thick, the ice-QLL and QLL-vapor interfaces
fluctuated independently, and were hardly distinguishable from those
of bulk water-vapor interfaces. sine-Gordon model of the solid-liquid
interface\cite{20,45}
 % End Summary 

 % Melting the ice one layer at a time, PNAS commentary (2017) 114, 2,
 % 195-197. Steal the citations from this paper. Seems like there are
 % good things to talk about from this paper to in relation to the
 % recent PNAS paper.

 % Surface Premelting of Ice, J.Phys.Chem.C. 111,27 (2007) 9631
\subsection{Structure of Ice Surface}
To begin to understand the surface premelting of ice, we must first
understand what this surface looks like. A wide variety of
experimental techniques have been used to determine the structure,
dynamics, properties, and width of this liquid-like surface film.
Materer \textit{et al.} studied an ultrathin ice surface exposing the
basal face grown on a Pt(11) substrate at 90K (well below complete
surface melt) using low-energy electron diffraction (LEED)
spectrocopy, in conjunction with total-energy calculations and
molecular dynamics simulations.\cite{Materer95,Materer97} Their results suggested
the basal face is terminated with a full bilayer of molecules, and not
a half bilayer as some had previously conjectured. The same surface
structure was observed later by NAME using helium atom scattering
experiments at 30K.\cite{Braun98,Glebov00}

Sum frequency generation (SFG) spectroscopy is able to probe the
orientational disorder of molecules at a surface. The basal surface of
ice I-$_\mathrm{h}$ is known to present dangling or 'free' O-H bonds
normal to the interface. The orientational order of these dangling
bonds can be probed using SFG spectroscopy, and observing the
temperature at which the intensity of the SFG signal decreases can
estimate the onset of the QLL. Name and Name have probed the basal
surface using this technique for temperatures betwen 173-271
K~.\cite{Wei01,Wei02} They observed that the disorder became apparent
as the temperature approaches 200 K~, and increased with warmer
temperatures.  These results indicate that the structural ordering
decreases in a continuous fashion from bulk ice to the liquid-like
layer, which also agree with results from proton channeling and
glancing-angle X-ray scattering.\cite{Golecki78,Dosch95} Using
molecular dynamics simulations, Ikeda-Fukazawa and Kawamura found
molecules present on the surface of a basal plane with large
vibrational and rotational motions at temperatures well below the bulk
melting temperature.\cite{Ikeda-Fukazawa04} Their results supported
the continous variation of structural ordering across the qausi-liquid
layer observed by the SFG spectroscopy.

\subsection{Thickness of the Premelting Layer}
Many surface science techniques have been used to quantify the
temperature dependence of the quasi-liquid layer
thickness.\cite{Kouchi87,Golecki78,Dosch95,Beaglehole80, Bluhm99,
  Bluhm02, Furukawa87, Elbaum93, Dosch96, Doppenschmidt00, Kaverin04,
  Lied94} While each unique technique probes a different property of
the surface, thus developing a unique metric for whether the QLL has
developed, there is common agreement that the onset temperature is
around 243 K~. However, since the techniques probe different aspects
of the surface, there is disagreement in the temperature dependence of
the thickenss of the QLL. Surface techniques which probe structural
changes tend to be more sensitive and usually suggest lower
temperatures for the onset of the premelting. Contrary to this,
techniques which probe bulk properties of the film require a larger
number of molecules for signal generation, and tend to yield warmer
temperatures for the onset. Due to these discrepencies, reported
values for the QLL width range over 2 orders of magnitude at a given
temperature.\cite{Rosenberg05,Dosch96}

Optical reflection experiments have shown that the temperature
dependence of the QLL thickness varies depending on the exposed
crystal facet.\cite{Elbaum93} While there has been considerable improvement
on the ability to grow crystals of ice exposing single
faces,\cite{Shultz14, Shultz17}, many experiments performed have studied
polycrystaline ice surface.

Even if one is able to grow a well behaved crystal of ice exposing a
single facet, surface contamination has been found to considerably
alter the surface properties and characteristics. J. Wettlaufer
concluded that the thickness of the QLL would be sensative to
impurities by METHOD HERE.\cite{Wettlaufer99} Similarly, M. Elbaum observed
promotion of surface melting when exposed to air via optical
reflection experiments.\cite{Elbaum93} 


\subsection{Friction of the Ice Surface}
Observed coefficients of friction on ice are about 1 order of
magnitude lower than found for other solids. However, these
coefficients are found to vary with many factors, including
temperature, sliding speeds, surface asparities of the contact
surface, and load.\cite{Evans76,Kietzig02,Persson01,Persson15} While
holding all other contributions constant, the coefficient of friction
is observed to decrease with increasing sliding speeds. At slow
sliding speeds, the friction is found to be very large. Investigating
glass and granite sliding on ice around $10^{-7}$, Persson observed
coefficients of friction of 0.3 and 0.9 respectively.\cite{Persson00} Li and
Somorjai attributed the discrepency in sliding speeds to be due to the
following; macroscopically at high sliding speeds frictional heating
causes surface melting, resulting in a lower coefficient of friction,
while microscopically the slider's surface asparities have less time
to push out the lubricating ice-melt and to cause large plastic
deformation on the ice surface.\cite{Li07} They proposed the
opposite to hold true at smaller sliding speeds, thus the observed
increase in friction with decreasing sliding speeds.

Name and Name have observed a decrease in friction with increasing
temperature.\cite{Evans76,Colbeck97}. This is commonly attributed to a
thicker QLL present on the surface of ice with increasing temperature,
where the thicker QLL acts as a better lubricating layer. However, the
optimum temperature for speed skating is well known to be around -7
degrees Celcius, with a reported coefficient of friction of
0.0046.\cite{Dekoning92} Above -7 degrees Celcius, the observed
friction is found to increase again, indicating that a plastic
deformation of the ice surface is causing an increase in the contact
area between the skate and the ice surface.\cite{Barnes66,Barnes71}
Since the load of the skater is not changing through this process, the
increase in the contact area yields a greater adhesive force and from
this the observed increase in friction.

\subsubsection{Atomic Force Microscopy Tip Penetration Measurements}
A quick intro of AFM...  Atomic force microscopy (AFM) is an
experimental technique in which an ultrafine metallic tip (with a
single atom at its point) is connected to a cantilever, and as the tip
approaches the surface, probes the surface topology by interpreting
the motion of the cantilever. BLAH. 

To probe the plastic deformation of ice surfaces, Pittenger \textit{et
  al.} have performed atomic force microscopy (AFM) experiments where
they indented the surface with the AFM tip.\cite{Butt00,
  Pittenger01,Bluhm00} They found yield strengths of 14 MPa at -10.4
degrees Celcius, and 20 MPa at -15.1 degrees Celcius, both of which
are smaller than Young's modulus of bulk ice. This indicates that the
ice surface undergoes plastic deformation during the indentation of
the tip. Pettinger also attempted to determine the mechanism for
plastic flow during the plastic deformation by performing indentation
experiments at varying tip approach speeds. Butt \textit{et al.}
argued the rate of pressurized melting due to the presence of the tip
controlled the tip penetration speed.\cite{Butt00} However, Pettinger
\textit{et al.} pointed out the pressure due to the tip was not large
enough to cause melting of the surface.\cite{Pittenger01} Therefore, at the
temperatures and pressures Pettinger performed the AFM experiments at,
one would expect a thick QLL to be present. Given that, the
penetration rate should be governed by the rate of flow viscosity of
the QLL out from under the tip. However, the viscosity of the QLL
layer is still unknown. These AFM experiments do indicate that the
viscosity of the QLL must be at least 2 orders of magnitue greater
than that of bulk supercooled water. Li and Somorjai conjecture that
the increase in viscosity could be due to the confinement of the QLL
between the AFM tip and the ice, or due to the intrinsic ordering of
the QLL itself.

\subsubsection{Lateral Force Microscopy Measurements}
Lateral force microscopy is an experimental technique in which ...

Using LFM, Bluhm \textit{et al.} have obtained friction coefficients
for ice grown on a mica substrate over a temperature range of -40 to
-24 degrees Celcius.\cite{Bluhm00} Plotting the lateral force against the
imposed load, they were able to extract a friction coefficient of
about 0.6, which was robust over the entire temperature range
spanned. Due to this, and that the value obtained is comparable to
static friction observed in macroscopic measurements, they believe the
QLL was pushed out from under the tip, resulting in the tip having
direct contact with the ice surface. (They observed a QLL of about 8 nm thick
on the surface at XX degrees Celcius.) 

\subsubsection{Viscosity of the QLL}
 Interpretation of AFM measurements of the friction of ice depends
 strongly on whether a QLL is present between the tip and the
 surface. However, since the viscosity of the QLL is not known, nor
 how the value changes with changing conditions such as temperature or
 confinement under the tip, many of the results reported above are
 open to conjecture. A promising technique to obtain the viscosity of
 the QLL is interfacial force microscopy (IFM). The benefit to IFM as
 compared to other AFM techniques, is it does not suffer from the
 'jump-in' instability as it is  force-controlled (more). Using IFM,
 measurements of the normal and lateral forces are able to be obtained
 throughout the interfacial separation.\cite{Joyce91} Recently, Name and
 Name have studied confined water between a Au(111) surface and a Au
 tip.\cite{Major06} The effective viscosity was observed to be 7 orders of
 magnitude greater than bulk water, with the increase attributed to
 the structural ordering required under confinement. 

 It would be interesting if an IFM investigation of the QLL on ice is
 able to charactarize the viscosity of the ice-melt, which might help
 to further explain the observed AFM results described
 above. Similarly, knowledge of the viscosity would also help
 ellucidate the mechanism for the observed QLL behavior at the leading
 and trailing edge of sliders.

 % End Summary


\subsection{Friction at Ice/Water Interfaces}
%%%%%%%%%%%%%%%%%%%%%%%%
% One anomoly is the reduced friction at ice/water interfaces, explain
% the slider problem. 
% Break down into the three contributions of friction, explain qll and role.
%
%%%%%%%%%%%%%%%%%%%%%%%%
% Ice/water interface
% ice/qll interface
%%%%%%%%%%%%%%%%%%%%%%%%
One anomoly of water in particular has been at the heart of a debate
amongst scientists for nearly two centuries; the observed low
coefficients of friction for sliding across ice.  Tribological studies of
ice in contact with a wide variety of materials (including other ice)
has been the focus of research for many years.  Approximately 150
years ago, Faraday attributed the freezing of two pieces of ice
together to be due to the ice surfaces being covered with a
quasi-liquid layer (QLL).\cite{Faraday1859} This marked the beginning
of the modern investigation on the properties of ice and the role the
surface liquid-like layer plays in ice friction. Soon thereafter,
Thomson incorrectly attempted to explain the presence of the
liquid-like layer as a result of pressure melting.\cite{Thomson1859}
Reynolds followed Thomson's work, and systematically investigated
sliding on ice. He also concluded that pressurized melting of the ice
surface was the governing physical process for the obeserved small
coefficients of friction.\cite{Reynolds1901} This view was widely
accepted, until Bowden and Hughes proposed that frictional heating may
primarily be responsible for the small coefficients of friction
observed for ice.\cite{Bowden1939}

Since these inaugural investigations of ice, a large and diverse
community of scientists has formed, studying ice and ice friction
for applications including ice skating and winter 
sports\cite{Rosenberg2005,Kietzig2010}, 
road safety and shoe soles\cite{Roberts1981,Higgins2008}, 
glacier movement\cite{Casassa1991, Sukhorukov2013, Pritchard2012},
and the fracture of the arctic sea 
ice\cite{Schulson2004,Weiss2007,Feltham2008,Lishman2011,Lishman2013}.  
Experimentally, the surface of ice has been probed by atomic force
microscopy
(AFM)\cite{Doppenschmidt1998,Bluhm1999,Bluhm2000}, scanning force
microscopy\cite{Bluhm1998}, ellipsometry\cite{Beaglehole1980,Beaglehole1993},
nuclear magnetic resonance (NMR)\cite{Ishizaki1996}, X-ray 
diffraction\cite{Dosch1996}, and photoelectron
spectroscopy\cite{Bluhm2002}. Further investigation has been performed by
computer simulations, studying bulk 
ice\cite{Kerr1988,Tse1988,Hayward1997,Gao2000,Rick2005,Dong2001,Weber1983,Wang2005,Kuo2005,Buch1998,Rick2001,Gay2002}, 
ice / vapor\cite{Kroes1992,Devlin1995,Ikeda-Fukazawa2004,Picaud2006,Conde2008,Pereyra2009},
and ice / water\cite{Baez1995,Bryk2002,Bryk2004,Bryk2004a,Gao2000,GarciaFernandez2006,Hayward2002,Hayward2001,Karim1988,Karim1987,Karim1990,Louden2013,Nada1997,NadaH.andFurukawa1995,Nada1996,Nada2000,Nada1997a} interfaces. 

Somewhere in here talk about antifreeze proteins?

Both experiments and computer simulations point towards the existence of a
QLL forming on the surface of ice at temperatures below
the bulk melting point. The formation of this layer is believe to be
driven by the termination of the periodic crystal structure at the
surface. The surface molecules are only weakly bound to their lattice
positions by the underlying ice, and with appreciable thermal energy
these molecules reorient (and at warmer temperatures translate along
the surface) to maximize their hydrogen bonds. This results in the
formation of the QLL, which is generally accepted as the reason why
ice displays a low friction 
coefficient\cite{Malenkov2009,Dash1995,Rosenberg2005,Dash2006}.


There have been
extensive investigations on ice friction, attempting to ellucidate the
roles of temperature\cite{Roberts1981,Higgins2008,Bowden1939,Evans1976,Derjaguin1988,Liang2003}, sliding
speed\cite{Evans1976,Derjaguin1988,Liang2003}, applied load\cite{Buhl2001,Bowden1939,Derjaguin1988,Baurle2006,Oksanen1982},
contact area\cite{Bowden1939,Baurle2007}, and
moisture\cite{Calabrese1980}. Recently, 
Kietzig \textit{et al.} have performed experiments consisting of sliding 
different steel alloy rings over a prepared ice surface.\cite{Kietzig2009} 
They investigated the effect of surface nanopatterning, hydrophobicity, and 
surface structure of the ice-exposed slider on the ice/slider friction. 
These properties were studied over a wide 
range of temperatures and sliding velocities. They found at all temperatures 
investigated, increasing the slider velocity with constant temperature, 
decreases the friction coefficient. After passing through a minimum, the 
friction coefficient slightly increases. This slight increase was attributed 
to added drag due to capillary bridges forming between the melt film and 
slider. They observed a decrease in friction with increasing temperature, 
up to a minimum at about $-4$ 
degrees Celsius. At temperatures warmer than this, there was an observed 
increase in friction which was attributed to capillary bridges and viscous 
shearing of the melt film. 
Through use of laser irradiation, the slider hydrophobicity was tuned
without changing the chemical nature of the material. Kietzig
showed that laser induced hydrophobicity resulted in fewer capillary 
bridges forming between the slider and the melt film. This reduced the amount
of viscous shearing of the ice-melt, resulting in a lower 
friction coefficient. 
%, resulting in a delayed onset of the increase 
%in friction coefficient with slider velocity.
While most investigations of ice friction focus on heterogeneous
materials\cite{Bowden1939,Evans1976,Derjaguin1988,Liang2003,Liang2005,Baurle2006,Baurle2007,Kietzig2009,Kietzig2010},
there have also been advances made on understanding ice-ice friction\cite{Oksanen1982,Kennedy2000,Maeno2004,Fortt2007,Fortt2011,Lishman2011,Samadashvili2013}.

From these studies, three distinct friction regiemes
have been found depending on the temperature and sliding velocity of the 
material; boundary friction, mixed friction, and hydrodynamic 
friction.\cite{Bhushan2002,Persson2015,Tuononen2016,Kietzig2009,Kietzig2010} 
Under each regieme, the observed 
friction is the result of a different physical process. In boundary friction, 
the lubricating layer of ice melt is only several molecules thick. 
This thin film is unable to support the sliding material's load, and friction
arises due to the surface asperities of the sliding material 
interacting with the ice surface.\cite{Bhushan2002} In the mixed friction 
regime, the ice melt lubricating layer is thicker than in the boundary regime,
but not yet sufficiently thick to maintain the 
sliding material's load. However, the ice melt film reduces solid-solid 
adhesion at the interface. This helps to alleviate some of the frictional 
forces, although the lubricating layer can form capillary bridges with the 
material, resulting in a drag force.\cite{Kietzig2009,Kietzig2010} If an 
ice melt lubricating layer is thick enough
to support the sliding material's load, the material's surface asperities are 
no longer in contact with the surface and the observed friction is primarily 
due to the capillary bridges formed between the ice melt and the material.
Under these conditions, the ice friction is classified as hydrodynamic 
friction.\cite{Kietzig2009,Kietzig2010} Thus the three regiemes are characterized by the 
extent a liquid-like layer of water mitigates the sliding material's load.
Through extension of contact melting theory, Fowler and Bejan have recently 
indicated that the lubricating ice melt film under the sliding material becomes 
thicker toward the trailing end.\cite{Fowler1993} Due to this, as a material 
slides over the surface of ice, the prevailing friction mechanism may be a 
combination of those found in each of the friction regiemes. 

In a recent review\cite{Kietzig2010}, Kietzig \textit{et al.} outlined 
popular experimental techniques used to investigate the coefficients of 
friction for a variety of materials sliding on ice, as well as their 
sensativity to temperature, slider load,
contact area, wettability and hydrophobicity of the slider, and many other
parameters. Of particular interest, the friction coefficients were found to 
increase with increasing slider velocity. This was attributed to three 
physical processes; adhesion forces of the
slider's asperities with the ice surface, breaking of capillary bridges 
between the slider and the ice surface, and the viscous shearing of the 
ice melt across the ice surface. While teasing apart the individual 
contributions has proven challenging, Kietzig\cite{Kietzig2009} and
Persson\cite{Persson2015,Tuononen2016} 
have made significant progress. However, there is still very little known about
water shearing over ice surfaces. Open questions include how the
structure of the interface changes 
during this process, and the role the presented crystal face plays on
the observed friction. 

 %%
% Describe general MD 
% Describe Transport Properties (TP)
% Describe MD approaches to obtaining TP
% Describe NEMD approaches to obtaining TP
% Describe RNEMD approaches to obtaining TP
%%
\subsection{Molecular Dynamics Simulations}
With a firm understanding of the problem which we turn to the method
we will use to investigate the problem. Molecular dynamics (MD)
simulations. In MD simulations, models describing atoms and molecules
are propogated by evolving Netwon's Second Law through time. We begin
by considering some atom with position $\vec{r}(t)$ at some time $t$,
and we Taylor-expand about this position.

Integrators, velocity verlet, shake, etc. Ensembles, calculation of
pressure and temperature. Leach had a great explanation of these.



\subsubsection{Water Models}
An investigation of the properties and dynamics of water anywhere in
the phase diagram cannot be conducted without first critically
analyzing the model proposed for the study. After XX years of
simulating water, we are still optimizing and searching for geometries
and distributions of charges and potential sites, mass sites, and many
other parameters which will reproduce the phase diagram of water in
closer detail. 

Include a table of parameters for common water models.
Include a table of properties estimated by water models, Tm, etc.

This is also important for understanding the Appendix about a water
model construction.

In this section we will discuss the different water models used in the
literature, and reasons why to use each one. 

\subsubsection{Transport Properties}
Transport phenomena are processes that describe the transfer (flux) of
mass, heat, momentum, and charge by molecular motions. These movements
have been well defined in both homogeneous and heterogeneous systems,
through bulk materials as well as across interfaces, and transport
coefficients have been defined relating the transfered quantities to
the phenomena.


\begin{longtable}{rcc}
	\caption{BALANCE AND CONSTITUTIVE EQUATIONS FOR MASS, ENERGY, AND MOMENTUM TRANSPORT}
	\label{tab:transport}
	\\\hline \hline
 	& \textbf{~~Balance Equations~~} & \textbf{~~Constitutive Equations~~}\\ \hline 
	\textbf{~~Mass~~} & $\frac{\partial c (\vec{r}, t)}{\partial t} + \nabla \vec{j} = 0$ & $\vec{j} = -D \cdot \nabla c(\vec{r}, t)$\\
	\textbf{~~Energy~~} & $C_p \frac{\partial T (\vec{r}, t)}{\partial t} + \nabla \vec{J} = 0$ & $\vec{J} = -\lambda \cdot nabla T(\vec{r}, t)$\\
	\textbf{~~Momentum~~} & $\rho \frac{D \vec{v}(\vec{r}, t)}{Dt} + \nabla \overleftrightarrow{\sigma} = 0$ & $\sigma_{x,z} = -\eta \cdot \nabla_z (\rho v_x)$\\ \hline \hline
\end{longtable}





Talk about shear viscosity and such.  In equilibrium molecular
dynamics simulations, the transport properties can be obtained by
utilizing the Green-Kubo relations. In this method, the fluctuations
about SOME PROPERTY are integerated over a long period of time until
convergence on the desired property is achieved. 

Due to the timescale of the simulations required, others have
approached the problem by performing non-equilibrium molecular
dynamics simulations. Here, the system is forced to be at some
non-equilibrium state, and the flux required to maintain that state is
calculated from the simulation. This process can be challenging,
however, simulation times are often shorter than those required for
using the Green-Kubo formalism. With the required flux in hand, the
transport property is calculated by using the relavent linear
constitutive relation. If the desired transport property is the
thermal conductivity ($\lambda$) of a material, two regions of the simulation box
will be thermostatted, and the kinetic energy flux ($J$) required to
maintain the temperature difference will be calculated. Once this
value is converged, $\lambda$ can be obtained by 
\begin{equation}\label{thermalTransport}
J_{z} = \lambda \big(\frac{\partial T}{\partial z}\big)
\end{equation}
where the separation has been taken to be along the $z$-dimension of
the system.  Here, $\big(\frac{\partial T}{\partial z}\big) $ is
determined by the temperature of the two regions and the distance the
regions are separated over. Similarly, the shear viscosity
($\eta$) of a fluid can be determined by forcing molecules at two
separated regions of the simulation box,
\begin{equation}\label{momentumTransport}
  j_{z}(p_{x}) = -\eta \big(\frac{\partial v_{x}}{\partial z}\big)
\end{equation}
where again the regions are taken to be separated along the
$z$-dimension of the simulation box.

\subsubsection{Reverse Non-equilibrium Molecular Dynamics}
While it is possible to obtain the transport properties of interest
through equilibrium or non-equilibrium molecular dyamics simulations,
Kuang and Gezelter have recently developed an approach to
non-equilibrium molecular dynamics in which they invert the measured
and imposed quantites. 


\subsection{Conclusion}
We began this chapter discussing the importance of water in our world
and in our lives. Everywhere we have found life, water has been at its
center, and many scientists believe that life would not be able to
exist without water. 

In the chapter to follow, I present my results for solid-liquid
friction at ice-I$_\mathrm{h}$/bulk water interfaces. We investigate
the tribological nature of four common low-energy facets often
presented to the surroundings. In this work we charactarized the
ice/water interface, defining and quantifying structural and dynamic
measures of interfacial width, and observe how the dynamics of water
molecules change as their local environment transitions from of bulk
water to that of bulk ice. We present a solid-liquid friction
coefficient that is well behaved in the negative slip boundary
condition, and report the first ever solid-liquid friction
coefficients for ice shearing through liquid water. We then show that
the observed friction is goverened by the density of solid-liquid
hydrogen bonds that form at the interface.

In Chapter 3, we present our results for the shear viscosity of the
QLL on the same four facets of ice.
