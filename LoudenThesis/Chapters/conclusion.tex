\chapter{CONCLUSIONS}\label{chap:Concl}

\begin{flushright}
\textit{``Everything not saved will be lost.''} \\
-Nintendo (1985) \\
\end{flushright}

The surface of ice is a complex and dynamic environment, molecular
motion and surface reconstruction is observed well below the bulk
melting temperature of the crystal. This surface premelt, commonly
referred to as a quasi-liquid layer (QLL), acts as a lubricating layer
between the ice crystal and a slider traversing the surface.  Three
distinct regimes of friction have been established depending on the
extent this QLL mitigates the interactions of the two solids.  The QLL
is observed to be highly sensitive to temperature, as well as slider
velocity. In the boundary and mixed friction regimes, the QLL is not
sufficiently thick to prevent direct ice / slider
interactions. Conversely, in the hydrodynamic friction regime the
slider is completely supported by the QLL, and the overall friction
can be described by the solid-liquid friction of the ice surface and
the quasi-liquid layer, the viscous shearing of the surface premelt,
and the drag imposed by the capillary bridges formed between the
quasi-liquid layer and the slider. While the third contribution is
sensitive to the chemical composition and surface structuring of the
slider, the first two components of hydrodynamic friction are quite
general.In this dissertation, the molecular origins of hydrodynamic
friction at the surface of ice-I$_\mathrm{h}$ are explored.

Molecular simulations of basal, prismatic, 14\degree~ pyramidal, and
secondary prismatic ice-I$_\mathrm{h}$ / water interfaces were
performed. The spatial transition between bulk liquid and ice was
characterized and quantified using both structural and dynamic order
parameters. For all order parameters investigated, the transition is
found to be continuous and estimates using dynamic measures give a
slightly broader interface as compared to the structurally determined
widths. The structure and dynamics of the interfacial liquid is found
to be distinctly different than that of the bulk liquid and ice. The
interface is observed to be robust to an imposed shear between the ice
crystal and the liquid. That is, no shear thinning or thickening was
observed. Decomposition of dynamic measurements show the short- and
long-time decay components behave differently closer to the interface.

The shearing of ice through water is observed to be in the stick
boundary condition. As such, an appropriate expression for
solid-liquid friction coefficients in this regime is presented, and
facet dependent friction is observed. We find the prismatic and
secondary prismatic surfaces exhibit the largest friction, the basal
surface presents moderate friction, and the pyramidal exhibits the
smallest. These friction coefficients are found to be invariant to the
shear rate and direction of the shear relative to surface
features. The observed friction coefficients are explained in terms of
the density of solid to liquid hydrogen bonds, which are identified
using the local tetrahedral ordering of the water molecules. A
momentum transmission model is proposed which relates the observed
friction with the density of solid to liquid hydrogen bonds, the shear
viscosity of the fluid, and the width of the interface. The
transmission model is robust under two different water potentials.

A temperature dependence study of the quasi-liquid layer at the basal
and prismatic surfaces of ice-I$_\mathrm{h}$ was also conducted. Using
density measures, the QLLs were found to be
$w~\sim~6.7$~\AA~wide. Spatially resolved one dimensional diffusion
constants showed anisotropic diffusion at the prismatic surface, and
isotropic diffusion at the basal surface. Spatially resolved reduced
viscosities for the QLLs were obtained, and in conjunction with a
temperature dependent model for bulk supercooled liquid water,
estimates of the shear viscosity of the QLLs were presented. These
results showed that the shear viscosity close to the prismatic ice /
QLL interface is much larger than that at the basal ice / QLL
interface, agreeing with our estimates of the solid-liquid friction
coefficients. Close to the QLL / vapor interface, the shear
viscosities come into agreement, and are observed to be between 1,000
and 10,000 times smaller than found close to the ice. We conclude,
that the low coefficients of friction commonly associated with ice
surfaces arises from the low shear viscosity of water molecules at
this interface. 

The work presented here encompasses two-thirds of the contributions to
hydrodynamic friction at ice surfaces. Further investigations are
necessary in order to determine the contribution of the capillary
bridges between the QLLs and a slider traversing the
surface. Decoupling the individual contributions of hydrodynamic
friction can prove challenging, however, progress is being
made. Recently, Professor Mary Jane Shultz has developed a method for
growing ice crystals which present a single facet. Dr. Mary Kietzig
has also made significant progress measuring friction coefficients
with varying slider speed and temperature, mapping out the friction
regimes. With these two methodologies, it is now possible to obtain
measurements of friction coefficients for single facet surfaces, as
opposed to the poly-ice surfaces commonly measured today. Together,
with the work I have presented here, a complete picture of
hydrodynamic friction may finally be achieved.
