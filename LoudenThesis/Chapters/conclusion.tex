\chapter{CONCLUSIONS}\label{chap:Concl}

\begin{flushright}
\textit{``Everything not saved will be lost.''} \\
-Nintendo (1985) \\
\end{flushright}

The surface of ice is a complex and dynamic environment, molecular
motion and surface reconstruction is observed well below the bulk
melting temperature of the crystal. This surface premelt
acts as a lubricating layer between the ice crystal and a slider
traversing the surface.  Three distinct regimes of friction are
observed depending on the temperature and slider velocity, 
boundary, mixed, and hydrodynamic. In this dissertation, the molecular
origins of hydrodynamic friction at the surface of ice-I$_\mathrm{h}$
are explored.

Hydrodynamic friction is comprised of three components, the
solid-liquid friction of the ice surface and the quasi-liquid layer,
the viscous shearing of the surface premelt, and the drag imposed by
the capillary bridges formed between the quasi-liquid layer and the
slider. While the third contribution is sensitive to the chemical
composition and surface structuring of the slider, the first two
components of hydrodynamic friction are quite general. 

Molecular simulations of basal, prismatic, 14\degree~ pyramidal, and
secondary prismatic ice-I$_\mathrm{h}$ / water interfaces were
performed. The spatial transition between bulk liquid and ice was
characterized and quantified using both structural and dynamic order
parameters. For all order parameters investigated, the transition is
found to be continuous and estimates using dynamic measures give a
slightly broader interface as compared to the structurally determined
widths. The structure and dynamics of the interfacial liquid is found
to be distinctly different than that of the bulk liquid and ice. The
interface is observed to be robust to an imposed shear between the ice
crystal and the liquid. That is, no shear thinning or thickening was
observed. Decomposition of dynamic measurements show the short- and
long-time decay components behave differently closer to the interface.

The shearing of ice through water is observed to be in the stick
boundary condition. As such, an appropriate expression for
solid-liquid friction coefficients in this regime is presented, and
facet dependent friction is observed. We find the prismatic and
secondary prismatic surfaces exhibit the largest friction, the basal
surface presents moderate friction, and the pyramidal exhibits the
smallest. These friction coefficients are found to be invariant to the
shear rate and direction of the shear relative to surface
features. The observed friction coefficients are explained in terms of
the density of solid to liquid hydrogen bonds, which are identified
using the local tetrahedral ordering of the water molecules. A
momentum transmission model is proposed which relates the observed
friction with the density of solid to liquid hydrogen bonds, the shear
viscosity of the fluid, and the width of the interface. The
transmission model is robust under two different water potentials.

A temperature dependence study of the quasi-liquid layer at the basal
and prismatic surfaces of ice-I$_\mathrm{h}$ was also
conducted. Spatially resolved diffusion coefficients computed normal
to the ice surface indicate a difference in the mobility of the
surface premelt on the two facets. Using the Stokes-Einstein relation,
estimates of the shear viscosity of the two quasi-liquid layers are
obtained. We find diffusion to be slower for water molecules within
the prismatic surface premelt, and correspondingly a higher shear
viscosity. The structure and dynamics of the quasi-liquid layers are
compared with simulations of supercooled bulk liquid at the same
temperatures. The surface premelting layers are found to behave
distinctly different as compared to the supercooled liquid, even
though structurally the two liquids are very similar.

The work presented here encompasses two-thirds of the contributions to
hydrodynamic friction at ice surfaces. Further investigations are
necessary in order to determine the contribution of the capillary
bridges between the surface premelt and a slider traversing the
surface. Decoupling the individual contributions of hydrodynamic
friction can prove challenging, however, progress is being
made. Recently, Professor Mary Jane Shultz has developed a method for
growing ice crystals which present a single facet. Dr. Mary Kietzig
has made significant progress measuring friction coefficients with
varying slider speed and temperature, mapping out the friction
regimes. It would be interesting if Professor Mary Jane Schultz
collaborated with Dr. Mary Kietzig to obtain measurements of friction
coefficients for single facet surfaces, as opposed to the poly-ice
surfaces commonly measured today. Together, with the work I have
presented here, a complete picture of hydrodynamic friction may
finally be achieved.
